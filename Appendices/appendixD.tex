\chapter{Command line arguments}\label{apd:commandline_arguments}

\begin{center}
\begin{longtable}{|p{4cm}|p{1cm}|p{5cm}|p{1.5cm}|p{1cm}|}
\caption{Table showing the command line arguments.} \label{tab:commandargs} \\

\hline \multicolumn{1}{|c|}{\textbf{Longhand}} & \multicolumn{1}{c|}{\textbf{Shorthand}} & \multicolumn{1}{c|}{\textbf{Description}} & \multicolumn{1}{c|}{\textbf{Type}} & \multicolumn{1}{c|}{\textbf{Default Value}} \\ \hline 
\endfirsthead

\multicolumn{3}{c}%
{{\bfseries \tablename\ \thetable{} -- continued from previous page}} \\
\hline \multicolumn{1}{|c|}{\textbf{Longhand}} & \multicolumn{1}{c|}{\textbf{Shorthand}} & \multicolumn{1}{c|}{\textbf{Description}} & \multicolumn{1}{c|}{\textbf{Type}}& \multicolumn{1}{c|}{\textbf{Default Value}} \\ \hline 
\endhead

\hline \multicolumn{5}{|r|}{{Continued on next page}} \\ \hline
\endfoot

\hline \hline
\endlastfoot

--input &
  -i &
  Path to the TSPLIB file that should be loaded in &
  String &
  None \\ \hline
--output &
  -o &
  Path to the folder where all the output data should be saved &
  String &
  None \\ \hline
--cluster\_type &
  -c &
  The clustering type to use, options are K\_MEANS, AFFINITY\_PROPAGATION, BIRCH, DBSCAN, OPTICS &
  String &
  None \\ \hline
--aco\_type &
  -aco &
  The ACO algorithm that will be used, options are ACO\_PY or ACO\_MULTITHREADED &
  String &
  None \\ \hline
--cluster\_tour\_type &
  -ctt &
  The algorithm to use to find the tours inside the clusters, options are ACO or GREEDY\_NEAREST\_NODE.  If ACO is used then the ACO algorithm specified in -aco is used &
  String &
  None \\ \hline
--numberclusters &
  -n &
  The number of clusters that should be created, this is only applied when the clustering algorithm requires the number of clusters to be set &
  int &
  20 \\ \hline
--ap\_convergence\_iter &
  -apci &
  The convergence\_iterations for the Affinity Propagation algorithm &
  int &
  500 \\ \hline
--ap\_max\_iter &
  -apmi &
  The max\_iterations value for the Affinity Propagation algorithm &
  int &
  2000 \\ \hline
--optics\_min\_samples &
  -oms &
  The min\_samples value for the Optics Algorithm &
  int &
  5 \\ \hline
--k\_means\_n\_init &
  -kmni &
  The n\_init value for the K\_Means algorithm &
  int &
  10 \\ \hline
--birch\_branching\_factor &
  -bbf &
  branching\_factor value for the Birch algorithm &
  int &
  50 \\ \hline
--birch\_threshold &
  -bt &
  threshold value for the Birch algorithm &
  float &
  0.5 \\ \hline
--dbscan\_eps &
  -de &
  eps value for the DBSCAN algorithm. &
  float &
  50 \\ \hline
--dbscan\_min\_samples &
  -dms &
  The min\_samples for the DBSCAN algorithm &
  int &
  3 \\ \hline
--automate\_dbscan\_eps &
  -ade &
  Should the DBSCAN algorithm eps value be automated &
  boolean &
  True \\ \hline
--aco\_alpha\_value &
  -aav &
  The alpha value for the ACO algorithm &
  float &
  1 \\ \hline
--aco\_beta\_value &
  -abv &
  The beta value for the ACO algorithm &
  float &
  3 \\ \hline
--aco\_rho\_value &
  -arv &
  The rho value for the ACO algororithm &
  float &
  0.03 \\ \hline
--aco\_q\_value &
  -aqv &
  The q value to use for the ACO algorithm &
  float &
  1000 \\ \hline
--aco\_ant\_count &
  -aac &
  The number of ants that the ACO algorithm should use &
  int &
  50 \\ \hline
--aco\_iterations &
  -ai &
  The number of iterations that the ACO algorithm should run for &
  int &
  100 \\ \hline
 &
  -run2opt &
  Whether 2-opt should be ran &
  boolean &
  True \\ \hline
--should\_cluster &
  -sc &
  Whether the clustering algorithm should be ran &
  boolean &
  True \\ \hline
 &
  -dpi &
  The DPI value that is used for the matplotlib graphs, a higher value makes the graphs larger and more detailed. &
  int &
  800 \\ \hline
--displayplots &
  -dp &
  Should the matplotlib plots be opened in a new window when created. Regardless of this setting they are always saved to the output folder &
  boolean &
  True \\ \hline
  --animate\_improvements &
  -animate &
  Should an animation be created showing the improvements that ACO and 2-opt make&
  boolean &
  True \\ \hline
\end{longtable}
\end{center}